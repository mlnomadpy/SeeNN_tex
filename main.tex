\documentclass[conf]{new-aiaa}
%\documentclass[journal]{new-aiaa} for journal papers
\usepackage[utf8]{inputenc}

\usepackage{graphicx}
\usepackage{amsmath}
\usepackage[version=4]{mhchem}
\usepackage{siunitx}
\usepackage{amssymb}
\usepackage{longtable,tabularx}
\usepackage{booktabs}

\usepackage{acro}
\usepackage{algorithm}
\usepackage{algpseudocode}
\usepackage{caption}
\usepackage{subcaption}


\usepackage{hyperref}
\usepackage{cleveref}
\usepackage{pgfplots}
\usepackage{tikz}
\usepackage{ifthen}
\usepackage{multirow}

\usetikzlibrary{matrix,calc}
% \addbibresource{bibs/ieee.bib}

\setlength\LTleft{0pt} 

\title{SeeNN: Leveraging Multimodal Deep Learning for In-Flight Long-Range Atmospheric Visibility Estimation in Aviation Safety}

\author{Taha Bouhsine \footnote{Graduate Research Fellow}, Giuseppina Carannante.\footnote{Postdoctoral Fellow}, Nidhal C. Bouaynaya.\footnote{Associate Dean for Research \& Graduate Studies and Professor of Electrical \& Computer Engineering}}
\affil{Electrical and Computer Engineering Department, Henery M.Rowan College of Engineering, Rowan University, Glassboro, New Jersey, 08028}
\author{Soufiane Idbraim \footnote{Computer Science Professor and Head of IRF-SIC Laboratory}}
\affil{IRF-SIC Laboratory, Computer Science Department, Faculty of Sciences Agadir, Ibn Zohr University, Agadir, Morocco}
\author{Phuong Tran, Grant Morfit, Maggie Mayfield, Charles Cliff Johnson}
\affil{William J. Hughes Technical Center, Federal Aviation Administration, Atlantic City, NJ, USA}

\begin{document}

\maketitle

\begin{abstract}
Accurate, real-time estimation of atmospheric visibility is a critical yet challenging task in aviation safety. While deep learning has shown promise, unimodal approaches relying solely on RGB imagery often fail to capture the complexity of atmospheric conditions, leading to limitations in reliability and accuracy. This paper introduces SeeNN, a novel multimodal deep learning framework designed for robust, long-range, in-flight visibility estimation. SeeNN integrates information from five diverse modalities: RGB imagery, depth maps, normal surface maps, edge maps, and entropy maps. To facilitate the development and evaluation of such models, we also present SeeSet V1, a new, comprehensive, and publicly available benchmark dataset featuring a wide range of altitudes, land covers, and visibility conditions. Our extensive experiments demonstrate the superiority of the multimodal approach. The SeeNN framework achieves a classification accuracy of over 97\%, a significant improvement upon the 87.92\% accuracy of a baseline unimodal RGB model. This work underscores the substantial potential of multimodal fusion to enhance the reliability of automated visibility estimation systems, representing a key advancement toward improving safety and operational efficiency in aviation and other domains where visibility is a critical factor.
\end{abstract}

% \section{Nomenclature}

% {\renewcommand\arraystretch{1.0}
% \noindent\begin{longtable*}{@{}l @{\quad=\quad} l@{}}
% $A$  & amplitude of oscillation \\
% $a$ &    cylinder diameter \\
% $C_p$& pressure coefficient \\
% $Cx$ & force coefficient in the \textit{x} direction \\
% $Cy$ & force coefficient in the \textit{y} direction \\
% c   & chord \\
% d$t$ & time step \\
% $Fx$ & $X$ component of the resultant pressure force acting on the vehicle \\
% $Fy$ & $Y$ component of the resultant pressure force acting on the vehicle \\
% $f, g$   & generic functions \\
% $h$  & height \\
% $i$  & time index during navigation \\
% $j$  & waypoint index \\
% $K$  & trailing-edge (TE) nondimensional angular deflection rate
% \end{longtable*}}


\section{Introduction}

Atmospheric visibility \cite{visibilitybook} is a critical factor in aviation safety \cite{Kulesa2003WEATHERAA, fultz2016fatal, long2022analysis, fujita1977analysis, ramee2021analysis}, directly impacting a pilot's ability to navigate and make critical decisions. The tragic accident involving professional basketball player Kobe Bryant in January 2020 highlights the severe consequences of visibility-related issues. The National Transportation Safety Board (NTSB) report concluded that the pilot's decision to continue visual flight rules (VFR) into instrument meteorological conditions (IMC) led to spatial disorientation and the subsequent crash. This incident underscores the critical need for accurate, real-time visibility estimation in flight.

Visibility estimation in aviation is particularly challenging due to several factors. Currently, pilots rely heavily on their prior knowledge of landmarks and terrain features to gauge visibility \cite{ahlstrom2019assessments}. This dependence on pre-existing knowledge makes automatic estimation complex, as systems must account for varying geographical contexts. Additionally, conditions such as flying inside clouds or encountering rapidly changing weather patterns further complicate the problem, requiring robust and adaptable solutions.

% In an era of rapidly advancing aviation technology and increasing air traffic, ensuring flight safety remains paramount. This paper introduces a groundbreaking multimodal approach to in-flight visibility estimation, leveraging both monocular RGB cameras and additional multimodal data to significantly enhance aviation safety across diverse weather conditions and flight scenarios.

While deep learning methods have shown great promise in solving complex problems, they face challenges such as overfitting, generalization issues, and potential biases. These limitations are particularly evident in single-modality approaches, especially those relying solely on RGB images for visibility estimation across diverse flight conditions. RGB data alone often fails to capture crucial atmospheric properties or account for factors like glare, low-light conditions, or rapid weather changes \cite{Bouhsine2022, AitOuadil2023, li_meteorological_2017, chaabani_estimating_2018, palvanov_dhcnn_2018, choi_automatic_2018, you_relative_2019, li_method_2019, outay_estimating_2021}.

% start your phrase with a what everytime

To address these challenges, multimodal deep learning techniques have emerged as a promising solution. By integrating information from diverse modalities, these approaches enhance model capabilities and overcome the limitations of single-modality systems \cite{liu2018learn, castanedo2013review, molino2021improved, blasch2021machine}. Each modality captures a distinct type of information, often inaccessible through a single modality approach, resulting in a more comprehensive understanding of the environment and more accurate predictions.

The significance of multimodal deep learning solutions in visibility estimation is increasingly recognized \cite{palvanov_visnet_2019, department_of_computer_science_chu_hai_college_of_higher_education_80_castle_peak_road_castle_peak_bay_tuen_mun_nt_hong_kong_meteorology_2020, AitOuadil2023, app11030997, Zhang2023, You2022, Chen2022, Wauben2016, Cheng2018, Zhou2021}. These models can overcome limitations that plague single-modality systems \cite{huang2021makesmultimodallearningbetter}. The integration of multimodal deep learning (Table~\ref{tab:literature_review}) significantly enhances the robustness, safety, and reliability of deep learning solutions, making them particularly well-suited for critical real-world applications where reliability is crucial \cite{6817512, liang2024foundationsmultisensoryartificialintelligence}.



\begin{table}[htbp]
\centering
\caption{Literature Review of Modalities Used in the Literature for On-Ground Atmospheric Visibility Estimation }
\footnotesize
\begin{center}
\begin{tabular}{|p{3cm}|c|c|c|c|c|c|c|}
\hline
\textbf{Modality} & \textbf{\cite{You2022}} & \textbf{\cite{Palvanov2019}} & \textbf{\cite{Zhang2023}} & \textbf{\cite{Chen2022}} & \textbf{\cite{Wauben2016}} & \textbf{\cite{Cheng2018}} & \textbf{\cite{Zhou2021}} \\
\hline
Depth Map &  & & X & & & & \\
\hline
\begin{tabular}[l]{@{}l@{}}Transmission Map\end{tabular}  & X & & X & & X & & \\
\hline
Disparity Map & X & & & & & &  \\
\hline
Entropy & & & & &  & X & X \\
\hline
Edge Detection & & & & & X & &  \\
\hline
\begin{tabular}[l]{@{}l@{}}Contrast Computation\end{tabular}   & & & & & X & & \\
\hline
\begin{tabular}[l]{@{}l@{}}Koschmieder Law\end{tabular} & &  & & & X & & X \\
\hline
FFT & & X & & & & & \\
\hline
Spectral Filter & & X & & & & & \\
\hline
\begin{tabular}[l]{@{}l@{}}Dark Channel\\ Prior\end{tabular} & & & & X & X & & X\\
\hline
\end{tabular}
\label{tab:literature_review}
\end{center}
\end{table}

Despite advances in deep learning for visibility estimation, there remains a significant gap in addressing in-flight visibility. This gap is largely attributable to the scarcity of comprehensive in-flight visibility datasets, which are crucial for training and validating deep learning models in aviation contexts \cite{AitOuadil2023}. Existing datasets concentrate on short-range visibility, offering a limited spectrum of sceneries, and land covers, and are mainly focused on ground-level atmospheric visibility. Moreover, they are predominantly confined to ground-level altitudes, neglecting the variability and complexity introduced by different elevation viewpoints \cite{AitOuadil2023}. This restriction in dataset diversity hampers the development of more universally applicable and robust in-flight visibility estimation models.

In this work, we propose a multimodal framework for training visibility estimation systems to enhance the accuracy, trustworthiness, and robustness of DL models for atmospheric visibility estimation. We demonstrate how integrating diverse modalities can significantly mitigate the limitations inherent in unimodal RGB approaches, paving the way for more versatile and reliable DL applications in fields where environmental variability is critical. We also address the gap in publicly available datasets by creating a comprehensive dataset, capturing the visibility degradation across different land covers and altitudes.



In detail, the contributions of this work are as follows:
\begin{itemize}
    \item We provide a meticulously curated dataset as a benchmark for visibility estimation and related challenges such as dehazing and visibility restoration \cite{gui2023comprehensive}. This dataset, named SeeSet V1, will be made available to the community alongside the code at \url{https://github.com/skywolfmo/seeNN-paper}. The data, acquired from the X-Plane 11 flight simulator, encompasses a wide array of images captured under varied visibility conditions and at multiple altitudes, ranging from ground level to 2,000 feet Above Ground Level (AGL).  
    The dataset's comprehensiveness, spanning a wide range of visibility scenarios at multiple altitudes, establishes a robust foundation for training and evaluating visibility estimation approaches and other in-flight visibility restoration and image dehazing methods.
    
    \item  We have developed a multimodality fusion framework for estimating atmospheric visibility. This framework is used to train and validate multimodality deep learning models. Our results demonstrate that the multimodality deep learning models outperform the single-modality RGB model in terms of accuracy.


\end{itemize}



\section{Related Work}
\label{sec:related_work}

The estimation of atmospheric visibility has been a persistent challenge in the fields of computer vision and remote sensing. Over the years, a diverse array of methods has been developed, spanning from traditional techniques rooted in physical models to contemporary deep learning-based approaches.

\subsection{Traditional Visibility Estimation}

Early methodologies for visibility estimation were predicated on physical models of atmospheric scattering, most notably Koschmieder's law \cite{koschmieder1924theorie}. These methods typically involve the estimation of atmospheric parameters from imagery by analyzing features such as contrast, color, and other handcrafted attributes. For instance, a prominent technique leverages the dark channel prior, which posits that in most non-sky regions of an image, at least one color channel will exhibit very low intensity \cite{he2010single}. Although these methods have demonstrated efficacy under specific conditions, they frequently falter when confronted with the complexity and variability inherent in real-world atmospheric phenomena.

\subsection{Deep Learning for Visibility Estimation}

The advent of deep learning has catalyzed a paradigm shift, establishing data-driven approaches as the state-of-the-art in numerous computer vision tasks, including the estimation of atmospheric visibility. These methods obviate the need for explicit physical models or handcrafted features by learning to estimate visibility directly from data.

\subsubsection{Unimodal Approaches}

Many of the initial deep learning models developed for visibility estimation were unimodal, relying exclusively on a single input modality, typically RGB images. While these models have achieved considerable success, they are often constrained by the inherent ambiguities present in RGB data. For example, distinguishing between atmospheric conditions such as fog, haze, and clouds based solely on color information can be challenging. Furthermore, factors like glare, low-light conditions, and rapid meteorological changes can significantly degrade the performance of such unimodal systems \cite{Bouhsine2022, AitOuadil2023}.

\subsubsection{Multimodal Approaches}

To surmount the limitations of unimodal strategies, researchers have increasingly gravitated towards multimodal deep learning. By integrating information from multiple sources, these models can construct a more comprehensive and robust representation of the scene, thereby yielding more accurate visibility estimations. As detailed in \cref{tab:literature_review}, a variety of modalities have been explored in the literature, including depth maps, transmission maps, and thermal imagery \cite{You2022, Palvanov2019, Zhang2023, Chen2022, Wauben2016, Cheng2018, Zhou2021}. A clear consensus has emerged within the field: multimodal approaches consistently outperform their unimodal counterparts, offering enhanced reliability and accuracy—attributes that are paramount for safety-critical applications such as aviation.

\subsection{Multimodal Fusion Strategies}

A critical design consideration in multimodal deep learning is the strategy employed for fusing information from different modalities. The point of fusion within the network architecture delineates three principal categories of methods:

\begin{itemize}
    \item \textbf{Early Fusion:} In this approach, the raw data from various modalities are concatenated at the input level. The resultant combined data is then processed by a single, unified network. Although straightforward to implement, this method may prove suboptimal if the modalities possess disparate characteristics, as the network might struggle to learn a coherent shared representation.

    \item \textbf{Intermediate Fusion:} This strategy entails processing each modality through a dedicated feature extractor and subsequently fusing the learned features at an intermediate layer of the network. This allows the model to learn modality-specific features prior to their combination, which can lead to superior performance compared to early fusion.

    \item \textbf{Late Fusion:} In late fusion, each modality is processed by a separate network, and the final predictions are amalgamated at the decision level. This can be accomplished through techniques such as averaging the outputs, employing a voting scheme, or training a separate model to combine the predictions. Late fusion offers a flexible approach that accommodates the use of distinct architectures for each modality, but it may not fully leverage the inter-modal correlations at earlier stages of processing.
\end{itemize}

The selection of an appropriate fusion strategy is contingent upon the specific application and the intrinsic nature of the modalities being fused. In this work, we conduct a comparative analysis of different fusion strategies to identify the most efficacious approach for in-flight visibility estimation.

\section{Methodology}
\label{sec:methodology}

This section details the methodology employed in this study. We introduce a novel framework that leverages multiple modalities for the development of atmospheric visibility estimation solutions. A key component of this work is the construction of a new dataset, SeeSet V1, which encompasses both ground-level and elevated altitude conditions, addressing a critical gap in existing resources.

\subsection{SeeSet V1 Dataset}
\label{sec:seeset}

To address the limitations of existing datasets and to encompass a broader range of real-world operational scenarios, we have developed a novel aerial imagery dataset designated SeeSet V1. This dataset has been meticulously curated to include dynamic views from multiple locations, capturing scenery from both ground-based and aerial perspectives.

This section provides a comprehensive description of the data collection and labeling procedures (\cref{data_collection}). In \cref{modalities}, we detail the techniques utilized to generate the supplementary image modalities.

\subsubsection{Dataset Collection Process}
\label{data_collection}

The generation of our synthetic dataset was accomplished using an FAA-approved flight simulator. The use of this advanced simulator enabled the systematic and controlled acquisition of images, showcasing a diverse range of viewpoints and visibility degradation levels. The data collection process, as depicted in Figure~\ref{fig:data_collection_process}, commenced at ground level. Visibility was incrementally increased in discrete steps, up to a maximum of 100 miles. Upon reaching this limit, the viewpoint's altitude was elevated, and the visibility was reset to zero. This iterative procedure was continued up to a maximum altitude of 2,000 feet Above Ground Level (AGL).


\begin{figure}[htbp]
\centerline{\includegraphics[width=250pt]{imgs/data_collection_pipeline.png}}
\caption{Automatic Dataset Collection Process using X-Plane 11}
\label{fig:data_collection_process}
\end{figure}
 

The collected images are automatically labeled into five discrete bins, each tailored to specific \href{https://www.faa.gov/air_traffic/publications/atpubs/aim_html/}{FAA requirements}. This categorization is based on visibility conditions and regulations relevant to both ground-based and aerial environments. The designated bins serve as the basis for the five labels utilized in training our DL models. 
We report the classes (bins) specifications and the corresponding counts in \cref{tab:vis_img_count}.


\begin{figure}
  \centering
  \begin{subfigure}[b]{0.15\textwidth}
    \includegraphics[width=\textwidth, trim={7.5cm 0cm 7.5cm 0cm},clip]{imgs/examples/exp_0_featuresMiles_0.12427454732996136_featuresM_200_features.png}
    \caption{< 0.5 mile}
    \label{subfig:bin0}
  \end{subfigure}
  \begin{subfigure}[b]{0.15\textwidth}
    \includegraphics[width=\textwidth, trim={7.5cm 0 7.5cm 0},clip]{imgs/examples/exp_0_featuresMiles_0.9320591049747102_featuresM_1500_features.png}
    \caption{(0.5, 1] miles}
    \label{subfig:bin1}
  \end{subfigure}
  % Add the subfigrow environment here
    \begin{subfigure}[b]{0.15\textwidth}
      \includegraphics[width=\textwidth, trim={7.5cm 0 7.5cm 0},clip]{imgs/examples/exp_0_featuresMiles_1.8951868467819106_featuresM_3050_features.png}
      \caption{(1, 3] miles}
      \label{subfig:bin2}
    \end{subfigure}
    \begin{subfigure}[b]{0.15\textwidth}
      \includegraphics[width=\textwidth, trim={7.5cm 0 7.5cm 0},clip]{imgs/examples/exp_0_featuresMiles_4.038922788223744_featuresM_6500_features.png}
      \caption{(3, 5] miles}
      \label{subfig:bin3}
    \end{subfigure}
    \begin{subfigure}[b]{0.15\textwidth}
      \includegraphics[width=\textwidth, trim={7.5cm 0 7.5cm 0},clip]{imgs/examples/exp_0_featuresMiles_46.1462462872979_featuresM_74265_features.png}
      \caption{> 5 miles}
      \label{subfig:bin4}
    \end{subfigure}
  
  \caption{The impact of visibility on the multiple modalities for the 6N7 Sealane 01 View. Each row shows one modality: RGB, edge map, entropy map, FFT magnitude, and dark channel prior. Each column refers to a visibility bin.}
  \label{fig:impact_vis_deg_features}
\end{figure}


\begin{table}[htbp]
\centering
\caption{Visibility Categories and Images Count}
\label{tab:vis_img_count}
\begin{tabular}{@{}lllr@{}}
\toprule
Category & Visibility in miles  & Visibility in meters & Count  \\
\midrule
4        & $\geq$ 5 miles             &     $\geq$ 8046.72m                 & 67002  \\
3        & 3 to 5 miles         &      4828.03m to  8046.72m        & 19584  \\
2        & 1 to 3 miles         &            1609.34m to 4828.03m         & 19648  \\
1        & 0.5 to 1 mile  &               804.672m to 1609.34m      & 4928  \\
0        & $\leq$ 0.5 mile   &     $\leq$ 804.672m                 & 4938  \\
\midrule
Total    &    &                      &  116100  \\
\bottomrule
\end{tabular}
\end{table}


\subsubsection{Modalities}
\label{modalities}

\textbf{Monocular Depth Estimation:}

Monocular depth maps were extracted using the Omnidata toolkit \cite{eftekhar2021omnidata, ranftl2021vision}. This toolkit provides a scalable and comprehensive method for depth estimation, which is essential for understanding the spatial arrangement of a scene. The resulting depth maps furnish a pixel-wise measurement of distance from the viewpoint, thereby facilitating an accurate representation of the three-dimensional scene structure.

It is important to note a specific limitation of the depth estimation models employed. The training methodology for these models involves masking the sky and exclusively considering the ground for depth estimation. This may present challenges for certain images within our dataset that were captured at varying altitudes.

\textbf{Normal Surface Estimation:}

In addition to depth maps, the Omnidata toolkit was also utilized for normal surface estimation \cite{eftekhar2021omnidata}. This modality provides information regarding the orientation of surfaces within the image, which is crucial for discerning the geometric properties of the scene. In contrast to the depth estimation model, the normal surface estimator considers both sky and ground details.

\textbf{Entropy Map:}

An image entropy map is incorporated as a modality to enhance the model's sensitivity to variations in visibility, particularly under low-visibility conditions. The entropy map quantifies the amount of information, or uncertainty, present in different regions of an image.

\textbf{Edge Detection:}

Edge detection serves as another key modality, particularly well-suited for long-range visibility scenarios where the delineation of objects and scene boundaries is critical. By highlighting the contours and edges within an image, this modality aids in defining shapes and structures, thereby providing a clearer distinction between different objects and features in the scene.


\begin{figure}
    \centering
% Mean of Dark Channel Prior vs Visibility
    \begin{subfigure}[b]{0.4\textwidth}
        \includegraphics[width=\textwidth]{imgs/edge_density_vs_visibility.png}
    
    \end{subfigure}
    % Mean of Dark Channel Prior vs Visibility
    \begin{subfigure}[b]{0.4\textwidth}
        \includegraphics[width=\textwidth]{imgs/entropy_vs_visibility.png}
    \end{subfigure}
    % Mean of Edge Density vs Visibility
    \begin{subfigure}[b]{0.4\textwidth}
    \includegraphics[width=\textwidth]{imgs/dark_channel_vs_visibility.png}
        
    \end{subfigure}
    % Mean of FFT Magnitude vs Visibility
    \begin{subfigure}[b]{0.4\textwidth}
        \includegraphics[width=\textwidth]{imgs/fft_magnitude_vs_visibility.png}
    \end{subfigure}
    \caption{Impact of Visibility Degradation on Edge Density (a), Entropy map (b), Dark Channel Prior (c), and FFT Magnitude (d) vs Visibility in Miles}
    \label{fig:mean_of_features}
\end{figure}

 In Figures~\ref{fig:impact_vis_deg_features} and \ref{fig:mean_of_features}, we illustrate the impact of visibility degradation on various modalities for the same scene. Each row displays a single modality, while each column corresponds to a specific visibility bin.

\subsection{Fusing Modalities}

In the literature, numerous methods have been proposed for the fusion of different modalities within multi-stream networks \cite{akkus2023multimodal, radford2021learning, jia2021scaling}. These methods range from the simple concatenation of input streams in the input space to more complex fusion strategies at various levels of the model architecture.

Early fusion \cite{huang_fusion_2020} involves concatenating or otherwise preprocessing all input streams in the input space. The combined data is then fed into a single feature extractor. While this method is the simplest to implement, it is often limited, as the feature extractor may learn to disregard some modalities, with the feature representation being dominated by a single modality.

Intermediate fusion \cite{huang_fusion_2020}, a widely adopted approach, involves feeding the different modalities into separate encoder layers before fusing the extracted embeddings. In this paradigm, the model learns to extract salient features from each modality before they are combined, thereby preventing any single modality from dominating the feature space. A significant advantage of this architecture is its compatibility with recent advancements in representation learning, such as contrastive learning or unsupervised representation learning, where fusion occurs between the encoder and decoder layers or at the initial stages of processing.

Late fusion \cite{huang_fusion_2020} represents another fusion strategy, wherein each modality is passed through its own complete network until the decision layer (e.g., a classifier). Fusion is then performed at the decision level, either through a voting mechanism between the different models or by averaging their respective outputs.

\subsubsection{Multimodal Fusion Methods}

Various techniques for multimodal fusion have been proposed in the literature, including Tensor Fusion \cite{zadeh2017tensorfusionnetworkmultimodal}, Low-Rank Fusion \cite{liu2018efficientlowrankmultimodalfusion}, and attention mechanisms \cite{NEURIPS2021_76ba9f56}. Although each method possesses its own set of advantages and disadvantages, self-attention has emerged as a foundational component for many recent large-scale models. While it typically requires a larger volume of training data, its computational cost is significantly lower compared to methods such as tensor fusion.

\subsubsection{The SeeNN Multimodal Fusion Framework}

\label{subsub:proposed}


\begin{figure}
    \centering
    \begin{subfigure}[b]{0.64\textwidth}
        \includegraphics[width=\textwidth]{imgs/SeeNN_Expanded.pdf}
    \caption{}
    \end{subfigure}
    \begin{subfigure}[b]{0.2\textwidth}
        \includegraphics[width=\textwidth]{imgs/Projection Head.pdf}
    \label{fig:projection_head}
        \caption{}
    \end{subfigure}
\caption{
(a) SeeNN Framework: The framework first extracts features (entropy map, surface normals map, edge map, depth map) from the input image. Separate encoders $\phi_{m}(\cdot)$ ($\phi_{m}(\cdot)$ denotes modality encoders) process these features followed by a projection head (b), followed by fusion of these features through a Connector and prediction via a classifier $\mathbf{\hat{y}}$.
(b) Projection Head: The input vector is transformed by an MLP (Multi-Layer Perceptron) with a non-linear activation function (GeLU) and dropout for regularization.
}
\label{fig:seeeNN}
\end{figure}

The proposed SeeNN framework, illustrated in Figure~\ref{fig:seeeNN}, integrates multimodal deep learning techniques to process images concurrently with multiple derived modalities.

Initially, each input RGB image \( I \) undergoes a series of transformations via modality estimators to generate a depth map \( E_d(I) \), a normal surface map \( E_n(I) \), an edge detection map \( E_e(I) \), and an entropy map \( E_s(I) \). Each of these modalities captures distinct characteristics of the input, providing a diverse set of perspectives on the image's content.

Let $m$ denote a specific modality (i.e., generated depth map $depth$, normal surface $normal$, entropy map $entropy$, edge map $edge$, and RGB image $rgb$). We employ different backbone models $\Phi_{m}(\cdot)$ for each modality input $X_m$. In this work, we utilize DenseNet121 \cite{huang_densely_2018} as the architecture for all $\Phi_{m}$. The resulting embedding from each encoder is fed to a projection head $P_m$, which consists of a Multi-Layer Perceptron (MLP) with a non-linear activation function (GeLU) and dropout for regularization. This is followed by a layer normalization step, which is crucial for aligning the feature representations and mitigating the risk of dominance by any single modality. This process yields a feature vector \( F_m \).

This procedure is applied to the RGB image $X_{rgb}$, depth map \( X_{depth} \), normal surface map \( X_{normal} \), entropy map \( X_{entropy} \), and edge map \( X_{edge} \) to obtain the feature vectors $F_{rgb}$, $F_{depth}$, $F_{normal}$, $F_{entropy}$, and $F_{edge}$, respectively.

Following the projection heads, the SeeNN framework concatenates these embeddings into a single, comprehensive feature vector \( F \). This concatenation is represented as \( F = [F_{rgb}, F_{depth}; F_{normal}; F_{entropy}; F_{edge}] \). This composite vector is then fed to a connector module, $C$, which is responsible for fusing these modalities.

Finally, an MLP classifier head is applied to the fused feature vector to obtain the final prediction, $\hat{y}$.

For the connector module, we explored two primary methods. The first method involves passing the flattened feature vector \( F \) directly to the MLP, representing a simple yet effective fusion of the different features. The second method utilizes an attention block to perform self-attention on \( F \), followed by flattening the output and feeding it to the MLP head.

\subsubsection{Experimental Setup}

For this study, we utilized our custom-collected dataset, SeeSet V1 (\ref{sec:seeset}), which comprises 320 distinct views collected across 20 locations with varying land covers, each with visibility ranging from 0 to 100 miles. The dataset was partitioned into training and validation subsets using a holdout approach. Specifically, all views from a predefined set of locations were reserved for the validation set, ensuring that the model does not overfit to specific sceneries and instead learns to estimate visibility based on image degradation \cite{Bouhsine2022}. This resulted in a training set of $100,350$ instances and a validation set of $15,750$ instances. All images in the dataset were preprocessed to an input resolution of $224 \times 224$ pixels.

We employed the Omnidata models to preprocess the RGB images and extract the estimated Depth Map and Normal Surface \cite{eftekhar2021omnidata}. This approach, based on the DPT-Hybrid architecture \cite{ranftl2021vision}, is analogous to methods used in the literature to generate pseudo-labels from RGB data for pre-training multimodal models \cite{bachmann2022multimae, wang2024largescale}.

For the other modalities, namely the edge map and the entropy map, the RGB images were processed through handcrafted estimators, as depicted in Figure~\ref{fig:seeeNN}.

All models were trained for 100 epochs using the Adam optimizer with a learning rate of $0.001$. A batch size of $32$ was used for all training procedures.

\section{Results and Discussion}
% Talk about resulted data
% compare the different models results
% talk about the confusion matrix, talk about how the model is miss classifying some of the of the images as labels next to the original label, not too far away

Our research categorizes atmospheric visibility into five distinct categories, as summarized in \cref{tab:vis_img_count}. This categorization is based on the visibility range in miles, which was set according to the requirements of the FAA. Resulting in a comprehensive dataset of 116,100 images spanning various altitudes, land covers, and sceneries.

The training of a single modality RGB model did not achieve high accuracy on the different classes of our problem, with an overall accuracy of 87.92\%. This demonstrates the limitations of such models, especially when tested on unseen views. Unlike previous works that were tested on a limited number of views and with a data split that might have caused data leakage between the training set and test set, our model may have given false results during evaluation as it had already seen a similar image during training. To prevent the same problem in our experiments, we used the holdout method as we mentioned in the Experimental Setup,  %TODO refer to the chapter
where we hide certain locations totally from the model during the training.



% \begin{figure}[htbp]
% \centerline{\includegraphics[width=250pt]{imgs/val_train_acc_rgb.png}}
% \caption{Training and Validation accuracy for RGB Model}
% \label{fig:rgb_model_training_history}
% \end{figure}

% \begin{figure}[htbp]
% \centerline{\includegraphics[width=250pt]{imgs/val_train_acc.png}}
% \caption{Training and Validation accuracy for RGB Model}
% \label{fig:rgb_model_training_history}
% \end{figure}


Unlike the single modality RGB model, results from the multimodality models (\cref{tab:res_table}, \cref{fig:conf_mats})  show a big improvement in the accuracy of prediction in the validation set, compared with 87.92\% from the RGB model, when we fuse different modalities, we notice a leap of 10\% in accuracy. For instance, When we combine the embedding extracted from the RGB model with the embedding extracted from the depth map, the RGB-Depth model achieves a high accuracy of 96.53\% using a simple concatenate connector.


\begin{table*}

\centering
\caption{Results comparison of using different modalities, }
\label{tab:res_table}
\begin{tabular}{@{}lccccccr@{}}
\toprule
Connector & RGB & Entropy & Edge & Depth &  Normal Surface  & \# trainable param & val acc.  \\
\midrule
Unimodality & \checkmark &  &  &  &   &7M&  87.92  \\
\midrule
\multirow{5}{*}{Concatenate}& \checkmark & \checkmark &  &  &   &14M&  96.4  \\
& \checkmark &   & \checkmark &  &   &14M& 96.53   \\
& \checkmark &  &  & \checkmark &   &14M&   \textbf{97.57}  \\
& \checkmark &  &  & \checkmark & \checkmark  &21M&   97.14  \\
& \checkmark & \checkmark & \checkmark  & \checkmark & \checkmark  & 38M & 96.3  \\
\midrule
\multirow{4}{*}{Self-Attention} & \checkmark &   & \checkmark &  &   &14M& 96.86  \\
& \checkmark &  &  & \checkmark &   &14M&   96.31  \\
& \checkmark &  &  & \checkmark & \checkmark  &21M&   97.47  \\
& \checkmark & \checkmark & \checkmark  & \checkmark & \checkmark  & 38M & \textbf{97.63}  \\

\bottomrule
\end{tabular}
\end{table*}
% End Table

% \begin{table*}

% \centering
% \caption{Results comparison of using different modalities, }
% \begin{tabular}{c|c|c|c|c|c|c|c|c}
% \hline
% \label{tab:res_table}
% Connector & RGB & Entropy & Edge & Depth &  Normal Surface & image size & \# trainable param & val acc.  \\
% \hline
% Single Modality & \checkmark &  &  &  &  &$224^2$&7M&  87.92  \\
% % &  & \checkmark &  &  &  &$224^2$&7M&  95.6  \\
% % &  &   & \checkmark &  &  &$224^2$&7M& 97.23  \\
% % & &  &  & \checkmark &  &$224^2$&7M&   81.49  \\
% % & &  &  &  & \checkmark &$224^2$&7M&   85,68  \\
% \hline

% \multirow{6}{*}{Concatenate}& \checkmark & \checkmark &  &  &  &$2 * 224^2$&14M&  96.4  \\
% & \checkmark &   & \checkmark &  &  &$2*224^2$&14M& 96.53   \\
% & \checkmark &  &  & \checkmark &  &$2*224^2$&14M&   \textbf{97.57}  \\
% % & \checkmark &  &  &  & \checkmark &$2*224^2$&14M&   In Training  \\
% & \checkmark &  &  & \checkmark & \checkmark &$3*224^2$&21M&   97.14  \\
% & \checkmark & \checkmark & \checkmark  & \checkmark & \checkmark &$5 * 224^2$& 38M & 96.3  \\
% \hline
% % \multirow{5}{*}{Self-Attention}& \checkmark & \checkmark &  &  &  &$2 * 224^2$&14M&  In-Training  \\
% \multirow{4}{*}{Self-Attention} & \checkmark &   & \checkmark &  &  &$2*224^2$&14M& 96.86  \\
% & \checkmark &  &  & \checkmark &  &$2*224^2$&14M&   96.31  \\
% % & \checkmark &  &  &  & \checkmark &$2*224^2$&14M&   In-Training  \\
% & \checkmark &  &  & \checkmark & \checkmark &$3*224^2$&21M&   97.47  \\
% & \checkmark & \checkmark & \checkmark  & \checkmark & \checkmark &$5 * 224^2$& 38M & \textbf{97.63}  \\

% \hline
% \end{tabular}
% \end{table*}
% % End Table


% Discussion
The confusion matrix and results table highlight the efficacy of multimodal deep learning in atmospheric visibility estimation for most of the categories (Figure~\ref{fig:conf_mats}). The overall performance is good, but the multimodality approach still struggles with label 3, which represents the visibility range from 3 to 5 miles. We've observed that all different combinations of modalities have difficulty with this class specifically, and most of them incorrectly predict it as class 4. Future work could focus on improving our solution for this specific class. The combination of RGB and Depth alone resulted in improved results in this specific class. This suggests that future research could focus on testing different modalities to achieve better results and reducing the number of modalities used to estimate visibility. This would help to reduce the computation cost of estimating visibility without negatively impacting the accuracy of the models.

\begin{figure}
    \centering
% Mean of Dark Channel Prior vs Visibility
    \begin{subfigure}[b]{0.4\textwidth}
    \include{chapters/confmats/all_attention}
    
    % \includegraphics[width=\textwidth]{imgs/dark_channel_vs_visibility.png}
    \end{subfigure}
    % Mean of Dark Channel Prior vs Visibility
    \begin{subfigure}[b]{0.4\textwidth}
        \include{chapters/confmats/all_concat}
    \end{subfigure}
    % Mean of Edge Density vs Visibility
    \begin{subfigure}[b]{0.4\textwidth}
        \include{chapters/confmats/rgb_depth_concatenate}    
    \end{subfigure}
    % Mean of FFT Magnitude vs Visibility
    \begin{subfigure}[b]{0.4\textwidth}
        \include{chapters/confmats/rgb_depth_self_attention}    
    \end{subfigure}
    \caption{Confusion Matrix of the top 4 multimodality models}
    \label{fig:conf_mats}
\end{figure}


Another perspective that should be considered when deciding which modality you want to use for the visibility estimation is the computation cost required to run the inference, while the model that merged all the available modalities gave us the best results, it still requires passing the input to all the modality estimators and then to the backbones, which makes it require more resources. and when planning to deploy such models, you will start facing the limits to what your hardware can give, so you need to consider that, especially for embedded devices.

While our dataset addressed the gap in the availability of publicly accessible multi-view datasets for atmospheric visibility, one of its limitations is the lack of diversity in landscapes and land covers. While experiments were conducted to ensure that there was no data leakage and that the model was tested on unseen locations, there is still a need to improve the quality of such a dataset by using the latest simulator technologies such as Microsoft Flight Simulator, X-Plane 12 that provides near real-world simulations for visibility degradation, which will help tackle the lack of available datasets.

Future research should focus on diversifying the dataset, incorporating a wider range of atmospheric conditions and scenarios. This expansion is not just about quantity, but also about variety, ensuring the SeeNN framework is tested against different visibility situations. Adding different land covers will enrich the dataset, making the model more adaptable to varying geographical locations and environmental conditions.

Another future work will be the use of the different pretraining techniques used in the literature to improve representation extracted from the different images, as similar to our work, most of the multimodal works make the use of the advancement of task-specific models to generate pseudo labels that are used to pre-train the model in an unsupervised manner, removing the need of requiring labeled data.

Furthermore, we found the need to understand how the different visibility degradations impact the quality of features extracted features by the different architectures, successfully understanding this point will help us in improving the trustworthiness of such models in real-world situations and not only reliable for sandbox situations.




\section{Conclusion}

In this work, we have presented a novel multimodal approach to atmospheric visibility estimation, focusing on challenging in-flight scenarios through the application of advanced deep-learning architectures. Our primary contributions are twofold:

\begin{itemize}
    \item We propose SeeNN, a multimodal fusion framework that integrates RGB imagery with entropy maps, edge maps, depth information, and normal surface maps. Through rigorous experimentation, we demonstrate that this multimodal approach significantly outperforms single-modality baselines, including traditional RGB-based models. The superior performance of SeeNN underscores the efficacy of leveraging diverse data modalities in addressing the complex task of visibility estimation.
    \item We introduce a comprehensive, open-source benchmark dataset for atmospheric visibility estimation. This dataset is distinguished by its diversity, encompassing a wide range of altitudes, land cover types, and visibility conditions. It represents a valuable resource for the research community, enabling robust evaluation and comparison of visibility estimation algorithms.
\end{itemize}


Our empirical results indicate that the proposed multimodal deep learning framework offers substantial improvements in estimation accuracy compared to the single-modality RGB method.  The release of our benchmark dataset addresses a critical gap in the field, providing a standardized platform for algorithm development and evaluation. We anticipate that this resource will facilitate rapid progress in the domain, spurring the development of increasingly sophisticated multimodal deep learning techniques for atmospheric visibility estimation.

Future work may explore the integration of additional modalities, the application of more advanced fusion techniques, or the extension of our approach to related problems in atmospheric science. Moreover, the potential for transfer learning and domain adaptation in this context remains an open and promising avenue for investigation.

In conclusion, our work contributes to the growing body of research at the intersection of deep learning and atmospheric science, offering both methodological advancements and resources for the broader research community. As the field continues to evolve, we believe that multimodal approaches like SeeNN will play an increasingly crucial role in addressing complex environmental perception tasks, with far-reaching implications for aviation safety and beyond.


\section*{Acknowledgments}
% This work was supported by the Federal Aviation Administration
\bibliography{sample}




\end{document}
