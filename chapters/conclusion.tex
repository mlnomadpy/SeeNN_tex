\section{Conclusion}
\label{sec:conclusion}

In this paper, we have presented a novel multimodal deep learning framework, SeeNN, for the estimation of atmospheric visibility in challenging in-flight scenarios. Our work makes two primary contributions to the field.

First, we introduced the SeeNN framework, which effectively fuses information from RGB imagery, entropy maps, edge maps, depth maps, and normal surface maps. Our extensive experimental results demonstrate that this multimodal approach significantly outperforms traditional unimodal models that rely solely on RGB data. The superior performance of SeeNN underscores the value of integrating diverse data modalities to overcome the inherent ambiguities and limitations of single-source systems, thereby achieving more accurate and reliable visibility estimation.

Second, we have developed and released a comprehensive, open-source benchmark dataset for atmospheric visibility estimation. This dataset, a key contribution of our work, is distinguished by its diversity, encompassing a wide range of altitudes, land cover types, and visibility conditions. It provides a much-needed resource for the research community, enabling the robust training, validation, and comparative evaluation of visibility estimation algorithms.

Our empirical results show that the proposed multimodal framework offers substantial improvements in estimation accuracy over unimodal RGB methods. The release of our benchmark dataset addresses a critical gap in the field, providing a standardized platform for future research and development. We anticipate that this resource will catalyze further innovation in the domain, spurring the development of increasingly sophisticated multimodal deep learning techniques for atmospheric visibility estimation.

Future research could explore several promising avenues, including the integration of additional sensor modalities, the investigation of more advanced fusion architectures, and the application of our framework to related problems in atmospheric science. Furthermore, the potential for leveraging transfer learning and domain adaptation techniques in this context remains a compelling area for future investigation.

In conclusion, this work contributes to the growing body of research at the intersection of deep learning and atmospheric science, offering both methodological advancements and a valuable resource to the research community. As the field continues to evolve, we believe that multimodal approaches, such as the one presented in this paper, will play an increasingly pivotal role in addressing complex environmental perception tasks, with far-reaching implications for aviation safety and other domains.

